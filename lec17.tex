\documentclass[11pt]{article}
\usepackage{url,amsmath,setspace,amssymb,amsthm,fullpage}
\usepackage{algorithm,algorithmic}

% Scribe template modified from original created by UC Berkeley's EECS department
%\DeclareMathOperator*{\argmax}{arg\,max}
\newcommand{\heading}[5]{
   \renewcommand{\thepage}{#1-\arabic{page}}
   \noindent
   \begin{center}
   \framebox[\textwidth]{
     \begin{minipage}{0.9\textwidth} \onehalfspacing
       {\bf STATS 710 -- Sequential Decision Making with mHealth Applications} \hfill #2

       {\centering \Large #5

       }\medskip

       {\it #3 \hfill #4}
     \end{minipage}
   }
   \end{center}
}

\newcommand{\scribe}[4]{\heading{#1}{#2}{Instructors:
Susan Murphy and Ambuj Tewari}{Scribe: #4}{Lecture #1: #3}}

\newtheorem{lemma}{Lemma}
\newtheorem{theorem}[lemma]{Theorem}
\newtheorem{proposition}[lemma]{Proposition}

\newtheorem{definition}{Definition}

\newcommand\reals{\mathbb{R}} % real numbers
\newcommand\E[1]{\mathbb{E}\left[ #1 \right]} %expectation

%% multiarmed bandits %%

\newcommand\actions{\mathcal{A}}

\bibliographystyle{alpha}

\begin{document}
\scribe{17}{Nov 10, 2016}{Reinforcement Learning with UCB: Redux}{Ian Fox}

\section{Recap}
The presentation is roughly (but not exactly) based on \cite{auer2005online}.

We define expected regret as follows:
$$ R_M(L,D,\Pi) = E[\sum_{j=1}^{M} V^2_{\pi^*}(S_0) - \sum_{j=1}^{M} V^2_{\pi^{j-1}}(S_0)]$$

The RL version of the UCB algorithm in a simplified setting is as follows:
\begin{algorithm}
	\caption{Simple RL UCB}
	\begin{algorithmic}[1]
		\FOR{j=1 \TO M}
		\STATE Learner sees $S_{j,0} = S_0$
		\FOR{t=0,1}
		\STATE Learner selects $A_{j,k}$ using $\pi^{j-1}$
		\STATE Learner sees $S_{j,t+1}$
		\STATE Learner receives $R_{j,t} = R(S_{j,t}, A_{j,t}, S_{j,t+1})$
		\ENDFOR
		\STATE Learner forms $\pi^j$
		\ENDFOR
	\end{algorithmic}
\end{algorithm}

We define the following statistics:
\begin{itemize}
	\item $N_{s,a}^{m-1} = \sum_{j=1}^{m-1} \sum_{t=0}^1 \mathbf{1}[S_{j,t} = s, A_{j,t} = a]$
	\item $\rho_{s,a}^{m-1} = \sum_{j=1}^{m-1} \sum_{t=0}^1 \mathbf{1}[S_{j,t} = s, A_{j,t} = a]R_{j,t}$
	\item $p_{s,a,s'}^{m-1} = \sum_{j=1}^{m-1} \sum_{t=0}^1 \mathbf{1}[S_{j,t} = s, A_{j,t} = a, S_{j,t+1} = s']$
	\item UCB average reward: $\tilde{r}_a^{m-1}(s) = \frac{\rho_{s,a}^{m-1}}{N_{s,a}^{m-1}} + \sqrt{\frac{c \log(m)}{N_{s,a}^{m-1}}}$
	\item UCB transition probability: $\tilde{p}_a^{m-1}(s,s') = \frac{p_{s,a,s'}^{m-1}}{N_{s,a}^{m-1}} + \sqrt{\frac{c \log(m)}{N_{s,a}^{m-1}}}$
\end{itemize}

The UCB algorithm forms two policies for each $j$, one per time step. The policies are generated as follows:
\begin{gather*}
\pi_1^{m-1}(s) = \argmax_a \tilde{r}_a^{m-1}(s) = \argmax_{\pi_1} \tilde{V}^{1, m-1}_{\pi_1}(s)  \\
\tilde{V}^{1, m-1}_{\pi_1}(s) = \tilde{r}_{\pi(s)}^{m-1}(s) \\
\pi_0^{m-1}(s) = \argmax_a \tilde{r}_a^{m-1}(s) + \sum_{s'} \tilde{p}_a^{\pi_1^{m-1}}(s, s') \tilde{r}_{\pi_1^{m-1}}^{m-1}(s') = \argmax_{\pi_0} \tilde{V}^{2, m-1}_{\pi_0, \pi_1^{m-1}}(s)\\
\tilde{V}^{2, m-1}_{\pi}(s) = \tilde{r}_{\pi(s)}^{m-1}(s) + \sum_{s'} \tilde{p}_{\pi_1(s)}^{\pi_1^{m-1}}(s, s') \tilde{r}_{\pi_1}^{m-1}(s')
\end{gather*}
where $\tilde{V}^{1, m-1}_{\pi_1}(s)$ is the value function when 1 time step remains, similarly $\tilde{V}^{2, m-1}_{\pi_0, \pi_1}(s)$ is the value when 2 time steps remain.

Optimal policies are defined:
\begin{gather*}
\pi_1^*(s) = \argmax_a E[R|S=s, A=a] \\
E[R|S=s, A=a] \triangleq V_{\pi^*}(s) \\
\pi_0^*(s) = \argmax_a E[R_0|S_0=s, A_0=a] +E[V^{1}_{\pi^*(s_1)}(s_1) | S_0 = s, A_0 = a]\\
= \argmax_a r_a(s) + \sum_{s'}p_a(s, s') V^{1}_{\pi^*(s)}(s')
\end{gather*}

Our goal is to calculate $\pi_0^*, \pi_1^*$ for which $E_{\pi_0, \pi_1}[R_0+R_1]$ is maximal. Note:
$$V_{\pi_0, \pi_1}^2(s) = r_{\pi_0(s)}(s) + \sum_{s'}p_{\pi_0(s)}(s, s') V^1_{\pi_1(s')}$$

\section{UCB Regret}

\begin{theorem}
	$R_M(L,D,\Pi) = O(\frac{\log(M)}{\Delta})$ where the gap parameter $\Delta = \min_{\pi \ne \pi^*} V^2_{\pi^*}(s_0) - V^2_{\pi}(s_0)$
\end{theorem}
\textit{Proof:} First, note that we can approach this RL problem like a simple MAB, treating each policy as an arm. Using this approach we achieve this regret bound, but with a constant proportional to $|A|^{|S|}$. The author of the paper claims they proved a bound polynomial in $|A|, |S|$, but did not explicitly state the proof. The proof given here doesn't achieve this polynomial bound.

The regret suffered is
\begin{gather*}
\sum_{m=1}^{M} V^2_{\pi^*}(s_0) - V^2_{\pi^{m-1}}(s_0) \\
= \sum_{m=1}^{M} (V^2_{\pi^*}(s_0) - V^2_{\pi^{m-1}}(s_0)) \mathbf{1}[A_{m,0} = \pi_0^{m-1}, A_{m,1} = \pi_1^{m-1}(s_{m-1})]\mathbf{1}[\tilde{V}^{2, m-1}_{\pi^{m-1}}(s_0)-\tilde{V}^{2, m-1}_{\pi^{*}}(s_0) \ge 0] \\
\le \sum_{\pi \ne \pi^*} \Delta_\pi \sum_{m=1}^{M} \mathbf{1}[A_{m,0} = \pi_0^{m-1}, A_{m-1} = \pi_1^{m-1}(s_{m-1})]\mathbf{1}[\tilde{V}^{2, m-1}_{\pi^{m-1}}(s_0)-\tilde{V}^{2, m-1}_{\pi^{*}}(s_0) \ge 0] \\
* \mathbf{1}[N^{m-1}_{s_0, \pi_0(s_0)} \ge l_{\pi_0}, N^{m-1}_{\pi_1} \ge l_{\pi_1}] + \sum_{\pi \ne \pi^*} \Delta_\pi l
\end{gather*}
where 
\begin{gather*}
N_{\pi_1}^{m-1} = \sum_{j=1}^{m-1} \mathbf{1}[\Delta_{j, 1} = \pi_1(S_{j,1})] \\
= \sum_{j=1}^{m-1} \mathbf{1}[A_{j,1} = \pi_1(S_{j,1}), S_{j,1} = 1] + \mathbf{1}[A_{j,1} = \pi_1(S_{j,1}), S_{j,1} = 0] \\
= N_{1, \pi_1(0)}^{m-1} + N_{0, \pi(0)}^{m-1}
\end{gather*}

To put some words to this, we're following the proof strategy for the MAB UCB regret bound. We're assuming that we mess up, using suboptimal policy $\pi$, at least $l_\pi$ times. We've manipulated the regret term to account for this. We define $$l_\pi \triangleq \frac{36*\log(M)}{\Delta^2_\pi}$$. Now we'll use these goofs to control the probability of messing up in the future, by controlling:
$$
\sum_{\pi \ne \pi^*} \Delta_\pi \sum_{m=1}^{M} \mathbf{1}[\tilde{V}^{2, m-1}_{\pi^{m-1}}(s_0)-\tilde{V}^{2, m-1}_{\pi^{*}}(s_0) \ge 0] \mathbf{1}[N^{m-1}_{s_0, \pi_0(s_0)} \ge l_{\pi_0}, N^{m-1}_{\pi_1} \ge l_{\pi_1}]
$$
We'll focus on $\mathbf{1}[\tilde{V}^{2, m-1}_{\pi^{m-1}}(s_0)-\tilde{V}^{2, m-1}_{\pi^{*}}(s_0) \ge 0]$
\begin{gather*}
\tilde{V}^{2, m-1}_{\pi^{m-1}}(s_0)-\tilde{V}^{2, m-1}_{\pi^{*}}(s_0) \\
 = \tilde{V}^{2, m-1}_{\pi^{m-1}}(s_0)-V_\pi^2(s_0) - (\tilde{V}^{2, m-1}_{\pi^{*}}(s_0) - V_{\pi^*}^2(s_0)) + V_\pi^2(s_0) - V_{\pi^*}^2(s_0)
\end{gather*}
so equivalently
\begin{gather*}
\mathbf{1}[\tilde{V}^{2, m-1}_{\pi^{m-1}}(s_0)-\tilde{V}^{2, m-1}_{\pi^{*}}(s_0)] \\
= \mathbf{1}[\tilde{V}^{2, m-1}_{\pi^{m-1}}(s_0)-V_\pi^2(s_0) - (\tilde{V}^{2, m-1}_{\pi^{*}}(s_0) - V_{\pi^*}^2(s_0)) + V_\pi^2(s_0) - V_{\pi^*}^2(s_0) \ge 0]
\end{gather*}
Recall that 
$$\mathbf{1}[A+B+C \ge 0] \le \mathbf{1}[A \ge 0] + \mathbf{1}[B \ge 0] +\mathbf{1}[C \ge 0]$$
Let's focus on $\tilde{V}^{2, m-1}_{\pi^{m-1}}(s_0)-V_\pi^2(s_0)$ and write things out according to our definitions (warning: things are getting messy):
\begin{gather*}
\tilde{V}^{2, m-1}_{\pi^{m-1}}(s_0)-V_\pi^2(s_0) = \\
\frac{\rho^{m-1}_{s_0, \pi_0(s_0)}}{N_{s_0, \pi_0(s_0)}} - r_{\pi_0(s_0)} + \sqrt{\frac{c \log(m)}{N_{s_0, \pi_0(s_0)}}} \\
+ \sum_s p_{\pi_0(s)}(s_0, s) ( \frac{\rho_{s_1, \pi_1(s)}}{N_{s, \pi_1(s_1)}} - r_{\pi_1(s)}(s)+ \sqrt{\frac{c \log(m)}{N_{s, \pi(s)}}}) \\ 
+ (\frac{\rho^{m-1}_{s_0, \pi_0(1)}}{N_{s_0, \pi_0}} - p_{\pi_0(s_0)}(s_0, 1) + \sqrt{\frac{c \log(m)}{N_{s_0, \pi_0(s_0)}}})(\tilde{r}_{\pi_1(1)}(1) - \tilde{r}_{\pi_1(0)}(0))^+ \\
+ (\frac{\rho^{m-1}_{s_0, \pi_0(0)}}{N_{s_0, \pi_0}} - p_{\pi_0(s_0)}(s_0, 0) + \sqrt{\frac{c \log(m)}{N_{s_0, \pi_0(s_0)}}})(\tilde{r}_{\pi_1(0)}(0) - \tilde{r}_{\pi_1(1)}(1))^+ \\
= \\ 
\frac{\rho^{m-1}_{s_0, \pi_0(s_0)}}{N_{s_0, \pi_0(s_0)}} - r_{\pi_0(s_0)} - \sqrt{\frac{c \log(m)}{N_{s_0, \pi_0(s_0)}}} \\
+ \sum_s p_{\pi_0(s)}(s_0, s) ( \frac{\rho_{s_1, \pi_1(s)}}{N_{s, \pi_1(s_1)}} - r_{\pi_1(s)}(s)+ \sqrt{\frac{c \log(m)}{N_{s, \pi(s)}}}) \\ 
+ (\frac{\rho^{m-1}_{s_0, \pi_0(1)}}{N_{s_0, \pi_0}} - p_{\pi_0(s_0)}(s_0, 1) - \sqrt{\frac{c \log(m)}{N_{s_0, \pi_0(s_0)}}})(\tilde{r}_{\pi_1(1)}(1) - \tilde{r}_{\pi_1(0)}(0))^+ \\
+ (\frac{\rho^{m-1}_{s_0, \pi_0(0)}}{N_{s_0, \pi_0}} - p_{\pi_0(s_0)}(s_0, 0) - \sqrt{\frac{c \log(m)}{N_{s_0, \pi_0(s_0)}}})(\tilde{r}_{\pi_1(0)}(0) - \tilde{r}_{\pi_1(1)}(1))^+ \\
+ 2 \sqrt{\frac{c \log(m)}{N_{s_0, \pi_0(s_0)}}} + 2 \sqrt{\frac{c \log(m)}{N_{s_0, \pi_0(s_0)}}}(\tilde{r}_{\pi_1(1)}(1) - \tilde{r}_{\pi_1(0)}(0))^+ +  2 \sqrt{\frac{c \log(m)}{N_{s_0, \pi_0(s_0)}}}(\tilde{r}_{\pi_1(1)}(0) - \tilde{r}_{\pi_1(0)}(1))^+
\end{gather*}
This is where we ran out of time. From here things get nice, except that we can't control the visitation rate (rare transitions), leading to the exponential blowup.
\bibliography{stats710}
\end{document}


