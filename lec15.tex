\documentclass[11pt]{article}
\usepackage{url,amsmath,setspace,amssymb,amsthm,fullpage}
\usepackage{algorithm,algorithmic}

% Scribe template modified from original created by UC Berkeley's EECS department

\newcommand{\heading}[5]{
   \renewcommand{\thepage}{#1-\arabic{page}}
   \noindent
   \begin{center}
   \framebox[\textwidth]{
     \begin{minipage}{0.9\textwidth} \onehalfspacing
       {\bf STATS 710 -- Seq. Dec. Making with mHealth Applications} \hfill #2

       {\centering \Large #5

       }\medskip

       {\it #3 \hfill #4}
     \end{minipage}
   }
   \end{center}
}

\newcommand{\scribe}[4]{\heading{#1}{#2}{Instructors:
Susan Murphy and Ambuj Tewari}{Scribe: #4}{Lecture #1: #3}}

\newtheorem{lemma}{Lemma}
\newtheorem{theorem}[lemma]{Theorem}
\newtheorem{proposition}[lemma]{Proposition}

\newtheorem{definition}{Definition}

\newcommand\reals{\mathbb{R}} % real numbers
\newcommand\E[1]{\mathbb{E}\left[ #1 \right]} %expectation

\bibliographystyle{alpha}

\begin{document}
\scribe{15}{Nov 03, 2016}{UCB Finite-Horizon Reinforcement Learning}{Nick Seewald}
\section{Recap from Lecture 14}
Consider the setting in which we observe $M$ episodes, each with finite horizon $T$, and wish to learn an optimal policy $\pi^*$. We saw that $\pi^*$ is vector-valued, $\pi^* = \left(\pi^*_{0}, \ldots, \pi^{*}_{T-1}\right)^\top$, and 
\[
	\pi^{*} = \argmax_{\pi\in\Pi} V_{\pi}^{T}(s),
\]
where 
\[
	V_{\pi}^{T}(s) = \mathbb{E}_{A\sim\pi} \left[\sum_{t=0}^{T} R_{j,t} \mid S_{j,0} = s \right]
\]
for an episode $j=1,\ldots, M$. Recall that the state-action $T$-tuples 
\[
	\left\{S_{j,0}, \left\{S_{j,1}^{a_{0}}\right\}_{a_{0} \in \actions}, \left\{S_{j,2}^{a_{1}}\right\}_{a_{1} \in \actions}, \ldots, \left\{S_{j,T}^{a_{T-1}}\right\}_{a_{T-1} \in \actions}\right\} \quad j=1,\ldots,M
\]
are i.i.d., so explicitly denoting dependence on the index $j$ is not strictly required.

In Lecture 14, we saw that 
\[
	\pi_{T-t}^{*}(s) := \argmax_{a} r_{a}(s) + \sum_{s'} p_{a}(s,s') V_{\pi^{*}_{T-t}}^{t-1}(s'),
\]
where $V_{\pi^*_{T-t}}^{t}(s) := \max_{a} r_{a}(s) + \sum_{s'} p_{a}(s,s') V_{\pi^{*}_{T-t}}^{t-1}(s')$.

\section{Auer's UCB Finite-Horizon Reinforcement Learning}
We follow Auer and Ortner and introduce a UCB-type algorithm for the reinforcement learning problem \cite{auer2005online}. We make several simplifications. First, we restrict the within-episode horizon to $T=2$, and consider binary state and action spaces, $\left|\actions\right| = \left|\states\right| = 2$. Let $M$ be the number of episodes. Denote by $\pi^{m} = \left(\pi_{0}^{m}, \pi_{1}^{m}\right)^\top$ the policy learned over the past $m-1$ episodes which is applied in the $m^{\text{th}}$ episode. $\pi^{m}$ is a function of the history through the $m-1$ prior episodes, $\history_{m-1} = \left\{S_{j,0}, A_{i,0}, S_{j,1}, A_{j,1}, S_{j,2} : i<m\right\}$.

Auer's modified UCB algorithm is of the following form:
\begin{algorithmic}[1]
	\FOR{$j=1$ \TO $M$}
		\STATE Learner sees $S_{j,0} = s_{0}$
		\FOR{$t=0$ \TO $T-1$}
			\STATE Learner selects $A_{j,t}$ using $\pi^{j}$
			\STATE Learner receives $S_{j,t+1}$
			\STATE Learner forms reward $R_{j,t} = R\left(S_{j,t}, A_{j,t}, S_{j,t+1}\right)$
		\ENDFOR
		\STATE Learner forms $\pi^{j+1}$ from $j$ prior episodes.
	\ENDFOR
\end{algorithmic}
Notice that the starting state $s_{0}$ in line 2 does not depend on $j$: the same starting state is used for each episode. The UCB part of the algorithm comes in line 8, when the learner forms $\pi^{j+1}$. 

Typically, we require $j=M_{0}$ episodes are run to ensure that we see every state-action pair $(s,a) \in \states\times\actions$. This is non-trivial, and makes an assumption that transition probabilities are bounded away from 0 and 1. We ignore this in the following.

\subsection{Quantities Needed for the Learning Algorithm}

We introduce three statistics:
\begin{equation}
	\label{eq:N_sa}
	N_{s,a}^{m-1} = \sum_{j=1}^{m-1} \sum_{t=0}^{1} \ind\left\{S_{j,t} = s, A_{j,t} = a\right\}
\end{equation}
is the number of observations of state-action pair $(s,a)$,
\begin{equation}
	\label{eq:rho_sa}
	\rho_{s,a}^{m-1} = \sum_{j=1}^{m-1}\sum_{t=0}^{1} \ind\left\{S_{j,t} = s, A_{j,t} = a\right\} R_{j,t}
\end{equation}
keeps a running sum of the rewards for $(s,a)$, and
\begin{equation}
	\label{eq:p_sas}
	p_{s,a,s'}^{m-1} = \sum_{j=1}^{m-1} \sum_{t=0}^{1} \ind\left\{S_{j,t} = s, A_{j,t} = a\right\} \ind\left\{S_{j,t+1} = s'\right\}
\end{equation}
is used to form transition probabilities. We now define the UCB versions of two quantities:
\begin{equation}
	\tilde{r}_{a}^{m-1}(s) = \frac{\rho_{s,a}^{m-1}}{N_{s,a}^{m-1}} + \sqrt{\frac{c\log m}{N_{s,a}^{m-1}}}
\end{equation}
is the UCB version of the conditional expectation of the immediate reward, and
\begin{equation}
	\tilde{p}_{a}(s,s') = \frac{p_{s,a,s'}^{m-1}}{N_{s,a}^{m-1}} + \sqrt{\frac{c\log m}{N_{s,a}^{m-1}}}
\end{equation}
is the na\"{i}ve UCB version of transition probabilities. We say na\"{i}ve since the $\tilde{p}_{a}(s,s')$ may not sum to 1. To account for this, Auer introduces the following transformation~\cite{auer2005online}:
\begin{equation}
	\label{eq:ucb-modified-transition-prob1}
	\begin{split}
		\tilde{p}_{a}^{\pi_{1}, m-1}(s, 1) &={} \ind\left\{\tilde{r}_{\pi_{1}(1)}(1) \geq \tilde{r}_{\pi_{1}(0)}(0)\right\} \min\left\{1, \tilde{p}_{a}(s,1)\right\} \\
		&\quad + \ind\left\{\tilde{r}_{\pi_{1}(1)}(1) < \tilde{r}_{\pi_{1}(0)}(0)\right\} \left(1 - \min\left\{1, \tilde{p}_{a}(s,0)\right\}\right).
	\end{split}
\end{equation}
For a move from state $s$ into state $0$, we have $\tilde{p}_{a}^{\pi, m-1}(s,0) = 1- \tilde{p}_{a}^{\pi, m-1}(s,1)$.
The motivation behind these transformations is to privilege in the UCB world the state that looks best in the next stage. Notice that these probabilities are dependent on the policy $\pi$.

\subsection{Regret Bound for the UCB Learning Algorithm}
Recall our earlier notation $\pi_{t}^{m}(s)$, used to denote the policy at time $t$ applied in the $m^\text{th}$ episode given state $s$. We define
\[
	\pi_{1}^{m-1}(s) = \argmax_{a} \tilde{r}_{a}^{m-1}(s) = \argmax_{\pi_{1}} \tilde{V}_{\pi}^{1, m-1}(s),
\]
where $\tilde{V}_{\pi}^{1, m-1}(s) = \tilde{r}_{\pi_{1}^{m-1}(s)}^{m-1}(s)$, and
\[
	\pi_{0}^{m-1}(s) = \argmax_{a} \tilde{r}_{a}^{m-1}(s) + \sum_{s'} \tilde{p}_{a}^{\pi_{1}, m-1}(s,s') \tilde{r}_{\pi_{1}^{m-1}(s')}^{m-1}.
\]
Define 
\[
	\tilde{V}_{\pi}^{2, m-1}(s) = \tilde{r}_{\pi_{0}(s)}^{m-1}(s) + \sum_{s'} \tilde{p}_{\pi_{0}(s)}^{\pi_{1}, m-1}(s,s') \tilde{r}_{\pi_{1}^{m-1}(s')}^{m-1}.
\]
The UCB algorithm uses $\pi_{1}^{m-1}(s)$ to select $A_{m,1}$ and $\pi_{0}^{m-1}(s)$ to select to $A_{m,0}$.

We define the expected regret at episode $m$ as 
\[
	\regret_{m}\left(L, D, \Pi\right) = \mathbb{E}\left[\sum_{j=1}^{m} V_{\pi^{*}}^{2}(s_{0}) - \sum_{j=1}^{m} V_{\pi^{j}}^{2}(s_{0}) \right].
\]
Susan proposes the following bound, which may be improved upon.
\begin{proposition}
	$\regret_{m}(L, D, \Pi) = O\left(\log(m) / \min\left\{\Delta_{1}, \min_{s}\Delta_{0}(s)\right\}^{2}\right)$, where $\Delta_{1} = \min_{\pi\neq\pi^{*}} V_{\pi^*}^{2}(s_{0}) - V_{\pi}^{2}(s_{0})$ and $\Delta_{0}(s) = \min_{\pi_{1}\neq \pi_{1}^{*}} V_{\pi_{1}^{*}}^{1}(s) - V_{\pi_{1}}^{1}(s)$.
\end{proposition}

\begin{proof}[Proof (first part plus sketch)] 
	We start by expanding the definition of regret.
	\begin{align*}
		\regret_{m} ={}& \sum_{m=1}^{M} V_{\pi^*}^{2}\left(s_{0}\right) - V_{\pi^{m-1}}^{2}\left(s_{0}\right) \\
		={}& \sum_{m=1}^{M} V_{\pi^*}^{2}(s_{0}) - V_{\pi_{0}^{m-1}\pi_{1}^{*}}^{2}(s_{0}) + V_{\pi_{0}^{m-1}\pi_{1}^{*}}^{2}(s_{0}) - V_{\pi^{m-1}}^{2}(s_{0}) \\
		={}& \sum_{m=1}^{M} \left(V_{\pi^*}^{2}(s_{0}) - V_{\pi_{0}^{m-1}\pi_{1}^{*}}^{2}(s_{0})\right) \mathbbm{1}\left\{A_{m,0} = \pi_{0}^{m-1}(s_{0})\right\} \\
		&{}+ \left(V_{\pi_{0}^{m-1}\pi_{1}^{*}}^{2}(s_{0}) - V_{\pi^{m-1}}^{2}(s_{0})\right) \mathbbm{1}\left\{A_{m,1} = \pi_{1}^{m-1}(S_{m,1})\right\}.
	\end{align*}
	Notice that $V_{\pi^*}^{2}(s_{0}) \geq V_{\pi_{0}\pi_{1}^{*}}(s_{0})$ for all $\pi_{0}$ and $V_{\pi_0\pi_{1}^{*}}(s_{0}) \geq V_{\pi_{0}\pi_{1}}^{2}(s_{0})$ for all $\pi_{0}, \pi_{1}$. Now, define 
	\[
		\tilde{\Delta}_{0} = \max_{\pi_{0}\neq \pi_{0}^{*}} V_{\pi^*}^{2}(s_{0}) - V_{\pi_0\pi_1^*}^{2}(s_{0})
	\]
	and 
	\[
		\tilde{\Delta}_{1} = \max_{\pi_{1} \neq \pi_{1}^{*}, \pi_{0}} V_{\pi_{0}\pi_{1}^{*}}^{2}(s_{0}) - V_{\pi_{0}\pi_{1}}^{2}(s_{0})
	\]
	so we have
	\begin{align*}
		\regret_{m} \leq{}& \tilde{\Delta}_{0} \sum_{m=1}^{M} \mathbbm{1}\left\{A_{m,0} = \pi_{0}^{m-1}(s_{0})\right\} + \tilde{\Delta}_{1} \sum_{m=1}^{M} \mathbbm{1}\left\{A_{m,1} = \pi_{1}^{m-1}(S_{m,1})\right\} \\
		\leq{}& \tilde{\Delta}_{0} \sum_{\pi_{0}\neq \pi_{0}^{*}}\sum_{m=1}^{M} \mathbbm{1}\left\{A_{m,0} = \pi_{0}^{m-1}(s_{0})\right\} + \tilde{\Delta}_{1} \sum_{\pi_{1}\neq\pi_{1}^{*}} \sum_{m=1}^{M} \sum_{s} \mathbbm{1}\left\{A_{m,1} = \pi_{1}^{m-1}(s), S_{m,1} = s\right\} \\
		\leq{}& \tilde{\Delta}_{0} \sum_{\pi_{0}\neq\pi_{0}^*} \left(\ell_{0} + \sum_{m=1}^{M} \ind\left\{A_{m,0} = \pi_{0}(s_{0}) \cap N_{s_{0}\pi_{0}(s_{0})}^{m-1} \geq \ell_{0}\right\}\right) \\
		&{}+ \tilde{\Delta}_{1} \sum_{\pi_{1} \neq \pi_{1}^{*}} \sum_{s} \left(\ell_{1} + \sum_{m=1}^{M} \ind\left\{A_{m,1} = \pi_{1}(s) \cup N_{s, \pi_{1}(s)}^{m-1} \geq \ell_{1}\right\}\right).
	\end{align*}
	The last inequality holds because, for example,
	\begin{align*}
		\sum_{m=1}^{M} \mathbbm{1}\left\{A_{m,0} = \pi_{0}^{m-1}(s_{0})\right\} ={}& \sum_{m=1}^{M} \mathbbm{1}\left\{A_{m,0} = \pi_{0}^{m-1}(s_{0}) \cup N_{s{0},\pi_{0}(s_{0})}^{m-1} \geq \ell_{0}\right\} \\
		&{}+ \sum_{m=1}^{M} \mathbbm{1}\left\{A_{m,0} = \pi_{0}^{m-1}(s_{0}) \cup N_{s{0},\pi_{0}(s_{0})}^{m-1} < \ell_{0}\right\},
	\end{align*}
	and the second sum is at most $\ell_{0}$. 
	
	The crux of this proof, which will be finished in a subsequent lecture, is ensuring we have enough data, i.e., enough observations of each state-action pair. We take $\ell_{1}\simeq 8\log M / \Delta_{1}$ and $\ell_{0} \simeq 8\log M / \Delta_{0}$. For the sums over $m$ of indicators, we take the expectation and show that they sum to a constant.
\end{proof}



\bibliography{stats710}
\end{document}


