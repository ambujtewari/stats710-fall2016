\documentclass[11pt]{article}
\usepackage{url,amsmath,setspace,amssymb,amsthm,fullpage}
\usepackage{algorithm,algorithmic}

% Scribe template modified from original created by UC Berkeley's EECS department

\newcommand{\heading}[5]{
   \renewcommand{\thepage}{#1-\arabic{page}}
   \noindent
   \begin{center}
   \framebox[\textwidth]{
     \begin{minipage}{0.9\textwidth} \onehalfspacing
       {\bf STATS 710 -- Seq. Dec. Making with mHealth Applications} \hfill #2

       {\centering \Large #5

       }\medskip

       {\it #3 \hfill #4}
     \end{minipage}
   }
   \end{center}
}

\newcommand{\scribe}[4]{\heading{#1}{#2}{Instructors:
Susan Murphy and Ambuj Tewari}{Scribe: #4}{Lecture #1: #3}}

\newtheorem{lemma}{Lemma}
\newtheorem{theorem}[lemma]{Theorem}
\newtheorem{proposition}[lemma]{Proposition}

\newtheorem{definition}{Definition}

\newcommand\reals{\mathbb{R}} % real numbers
\newcommand\E[1]{\mathbb{E}\left[ #1 \right]} %expectation

\bibliographystyle{alpha}

\begin{document}
\scribe{1}{Sep 06, 2016}{Introduction}{Your name}

\section{Sequential Decision Making in mHealth}

This course will focus on the use of sequential decision making algorithms in mobile health applications to enable long term health behavior change and maintenance.
Behavior change can lead to improved health outcomes in many health conditions including cardiovascular diseases, substance abuse, and mental illnesses.
The promise of mobile health is that tailored interventions can be delivered to users at the right time, in the right context, as they go about living their daily lives.

The basic sequential decision making protocol in mobile health is as follows.

\renewcommand{\algorithmicfor}{\textbf{at}}
\renewcommand{\algorithmicendfor}{\textbf{done}}
\begin{algorithmic}[1]
\FOR{a given decision point}
\STATE mobile phone collects tailoring variables (the context) 
\STATE a decision rule (or policy) maps the tailoring variables into an intervention option (the action) 
\STATE mobile phone records the proximal outcome (interpreted as a reward, so higher is better) 
\ENDFOR
\end{algorithmic}
\renewcommand{\algorithmicfor}{\textbf{for}}
\renewcommand{\algorithmicendfor}{\algorithmicend\ \algorithmicfor}

For example, in a project called HeartSteps that aims to maintain physical activity in user after cardiac rehab, context variables include GPS location, outside weather, label (such as walking, running, in vehicle)
from google activity recognition, calendar busyness. There are two intervention options for momentary suggestions that are delivered on the smartphone's lock screen: $1$, i.e., provide a suggestion and 
$0$, i.e., don't provide a suggestion. The proximal outcome or reward is the number of steps walked (measured via a wristband) in a specified time period, say an hour, following the suggestion.

In this course we will look at increasingly sophisticated methods to sequentially choose actions to maximize sum of received rewards. The first class of methods, called {\em multi-armed bandits}, apply in situations
where there is no contextual information available to guide action selection. This is not a realistic case in mHealth application but serves as a foundation for more sophisticated methods. 
The second class of methods, called {\em contextual bandits}, do take into account contextual information but do not reason about delayed effects of actions on future contexts. Delayed effects can easily
occur in mHealth applications. E.g., providing a too many suggestions to the user can affect the burden involved in using the app and reduce effectiveness of future suggestions since they might be ignored by the user.
Finally we will look at general {\em reinforcement learning} algorithms that do take into account delayed effects.

\section{Multiarmed Bandits}

The name ``bandits" comes from ``one-armed bandit" which refers a slot machine in casinos operated by pulling a long arm at the side. These machines rob you of your money and hence the name ``bandits"!
Now imagine having several arms to choose from. You do not know which arms are good and which ones are bad. So you will have \emph{explore} new arms in the hope of finding good ones that have not been used much
so far but also \emph{exploit} your knowledge from previous rounds to stick to ones that have performed well in the past.

Robbins \cite{robbins1952some} originally formalized the problem that we now refer to as multiarmed bandit. We will use ``action" and ``arm" interchangeably. He considered the case of two actions
but it is not too difficult to extend the framework to consider a set of $K$ actions $\actions = \{0,1,\ldots,K-1\}$. To each action $a \in \actions$, we attach a distribution $D_a$. Assume that $D_a$ has finite mean $\mu_a$ and
consider a sequence of iid random variables
\[
R^a_1,R^a_2,\ldots
\]
distributed according to $D_a$.

The multiarmed bandit (MAB) protocol is as follows.

\begin{algorithmic}[1]
\FOR{$t=1$ to $T$}
\STATE Learner selects action $A_t \in \actions$
\STATE Learner receives reward $R_t = R^{A_t}_t$
\ENDFOR
\end{algorithmic}

\bibliography{stats710}

\end{document}


